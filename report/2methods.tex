% TODO: intro

\subsection{Markov Chain Monte Carlo}

The goal is now to extract information from the distribution defined by \eqref{eq:part_sum}. Expectation values of observables can be sampled using a Monte Carlo approach. Suppose we sample $N$ triangulations $T^{(i)}$ from \eqref{eq:part_sum}. The expectation value of an observable $f(T)$ can then in theory be obtained from
\begin{equation}
    \ev{f(T)}
    =
    \lim_{N \to \infty} \frac{1}{N} \sum_{i = 1}^N f(T^{(i)})
    .
\end{equation}
Directly sampling triangulations can be very difficult. Instead, we use a Markov Chain to explore the space of all possible triangulations. Hence the label Markov Chain Monte Carlo (MCMC) for our approach.

However, sampling from \eqref{eq:part_sum} is not very practical. It turns out that $\Omega(n) \sim n! 2^n$. This means that for $\lambda > \ln 2$ large volumes are suppressed so that a typical triangulation will contain only a handful of triangles.
However, when $\lambda < \ln 2$ the partition sum \eqref{eq:part_sum} diverges and the problem is ill-defined.
A solution is to consider only triangulations of a certain fixed volume at a time.
An advantage of this approach is that the desired distribution becomes uniform, since the weight of each triangulation depends only on its volume.
In practice, this can be obtained by using Markov Chain updates that keep the number of triangles fixed.

\subsection{Update rules}
% Should we also include alternative update rules we attempted? And why the don't work? It does show the amount of work and consideration we put into constcuting an effective MCMC simulation, but Im not sure.
\paragraph{Ergodicity}
An important property of a triangulation is how its volume is distributed over different timeslices. It therefore makes sense to choose an update rule that can change the length $\ell(t)$ at some timeslice $t$. However, because we wish to fix the volume this change in length should be compensated elsewhere. Thus we choose a move that takes an edge and accompanying triangles (we refer to such objects as "shards") from some source location, and moves it to some destination. This "shard move" is shown schemetically in Fig. \ref{fig:shard_move}. However, this move alone is not enough to satisfy ergodicity. Thus we introduce another move that can flip the diagonal between two triangles. This "flip move" is shown schemetically in Fig. \ref{fig:flip_move}. According to \cite{2012}, this combination of update rules is ergodic.

\begin{figure}
    \centering
    \begin{tikzpicture}

        % initial state for source
        % timeslices
        \draw (0, 0) -- (4, 0);
        \draw (0, 1) -- (4, 1);
        \draw (0, 2) -- (4, 2);

        % timelike connections
        \draw (0, 0) -- (1, 1) -- (0, 2);
        \draw (2, 0) -- (1, 1) -- (2, 2);
        \draw (2, 0) -- (3, 1) -- (2, 2);
        \draw (4, 0) -- (3, 1) -- (4, 2);

        % shard
        \draw[triangle = {red}] (2, 0) -- (3, 1) -- (2, 1) -- cycle;
        \draw[triangle = {red}] (2, 1) -- (3, 1) -- (2, 2) -- cycle;

        % order 4 vertex
        \node[order4] at (2, 1){};

        % initial state for destination
        % timeslices
        \draw (0, 3) -- (4, 3);
        \draw (0, 4) -- (4, 4);
        \draw (0, 5) -- (4, 5);

        % timelike connections
        \draw (0, 3) -- (1, 4) -- (0, 5);
        \draw (2, 3) -- (1, 4) -- (2, 5);
        \draw (2, 3) -- (3, 4) -- (2, 5);
        \draw (4, 3) -- (3, 4) -- (4, 5);

        % crack
        \draw[edge = {blue}] (2, 3) -- (3, 4) -- (2, 5);

        % final state for source
        % timeslices
        \draw (6, 0) -- (10, 0);
        \draw (6, 1) -- (10, 1);
        \draw (6, 2) -- (10, 2);

        % timelike connections
        \draw (6, 0) -- (7, 1) -- (6, 2);
        \draw (8, 0) -- (7, 1) -- (8, 2);
        \draw (8, 0) -- (9, 1) -- (8, 2);
        \draw (10, 0) -- (9, 1) -- (10, 2);

        % crack
        \draw[edge = {red}] (8, 0) -- (9, 1) -- (8, 2);

        % final state for destination
        % timeslices
        \draw (6, 3) -- (10, 3);
        \draw (6, 4) -- (10, 4);
        \draw (6, 5) -- (10, 5);

        % timelike connections
        \draw (6, 3) -- (7, 4) -- (6, 5);
        \draw (8, 3) -- (7, 4) -- (8, 5);
        \draw (8, 3) -- (9, 4) -- (8, 5);
        \draw (10, 3) -- (9, 4) -- (10, 5);

        % shard
        \draw[triangle = {blue}] (8, 3) -- (9, 4) -- (8, 4) -- cycle;
        \draw[triangle = {blue}] (8, 4) -- (9, 4) -- (8, 5) -- cycle;

        % order 4 vertex
        \node[order4] at (8, 4){};

        % arrow
        \draw[->, ultra thick] (4.5, 2.5) -- (5.5, 2.5);

    \end{tikzpicture}
    \caption{Shard move update rule. Shard source in red (bottom), destination in blue (top). Initial (left) and final (right) states. Order 4 vertices affected by this move are marked with a green dot.}
    \label{fig:shard_move}
\end{figure}

\begin{figure}
    \centering
    \begin{tikzpicture}
        % initial state
        % timeslices
        \draw (0, 0) -- (3, 0);
        \draw (0, 1) -- (3, 1);

        % timelijke connections
        \draw (1, 0) -- (0, 1);
        \draw (1, 0) -- (1, 1);
        \draw (2, 0) -- (2, 1);
        \draw (2, 0) -- (3, 1);

        % diagonal
        \draw[edge = {blue}] (1, 0) -- (2, 1);

        % maybe order 4 vertices
        \node[maybe_order4] at (1, 1){};
        \node[maybe_order4] at (2, 0){};

        % final state
        % timeslices
        \draw (5, 0) -- (8, 0);
        \draw (5, 1) -- (8, 1);

        % timelijke connections
        \draw (6, 0) -- (5, 1);
        \draw (6, 0) -- (6, 1);
        \draw (7, 0) -- (7, 1);
        \draw (7, 0) -- (8, 1);

        % diagonal
        \draw[edge = {blue}] (6, 1) -- (7, 0);

        % maybe order 4 vertices
        \node[maybe_order4] at (6, 0){};
        \node[maybe_order4] at (7, 1){};

        % arrow
        \draw[<->, ultra thick] (3.5, 0.5) -- (4.5, 0.5);
    \end{tikzpicture}
    \caption{Flip move update rule. The flipped edge is shown in blue. Order 4 vertices (if they exist) are marked with a green circle.}
    \label{fig:flip_move}
\end{figure}

\paragraph{Detailed Balance}
But this is not the whole story. Using these update rules we can now sample all triangulations of a given volume, but can we do so \emph{uniformly}, as required by \eqref{eq:part_sum}? A common way to ensure this is to enforce detailed balance. Since in this case the desired distribution is uniform, this amounts to enforcing
\begin{equation}
    p(T \to T') = p(T' \to T),
\end{equation}
for any two triangulations $T$ and $T'$. Here $p(T \to T')$ is the probability to go from $T$ to $T'$ in a single Markov Chain step.

First consider the shard move. In order to be able to remove a shard, we need a vertex of order 4 (i.e.: a vertex with 4 edges)\footnote{If you are worried that such a vertex might not always exist, you are correct. However if this is the case we can simply perform the flip move instead.}. Start by uniformly choosing such a vertex. To be consistent, we always remove the shard to the right of the selected vertex. Then we uniformly choose a spacelike edge (which is not the edge in the shard) and insert the shard to the left of it. This is done in such a way that an order 4 vertex is created between the selected and inserted edges (see Fig. \ref{fig:shard_move}). The probability of performing a particular instance of this move is then\footnote{Note that here $T$ and $T'$ must be two triangulations that are related by a application of a single shard move. If they are not, the probabilaties vanish in both directions and detailed balance is satisfied automatically.}
\begin{equation}
    p_{\text{shard}}(T \to T') = \frac{1}{V_4 (n/2 - 1)},
\end{equation}
with $V_4$ the number of order 4 vertices, and $n$ the number of triangles. Note that this move preserves $V_4$: One order 4 vertex is always destroyed where the shard is taken, and one is always created where it is inserted. Any surrounding vertices are not affected. Since $n$ is also constant, the inverse move will have identical probability and detailed balance is satisfied.

Now consider the flip move. This move is possible where neighbouring triangles have opposite orientation. First uniformly choose a triangle. To be definite, consider its neighbour to the right. If these two triangles have opposite orientation, perform the flip. If this is not the case, reject the move but still count it as a step in the Markov Chain. The probability of a particular move of this type is
\begin{equation}
    p_{\text{flip}}(T \to T') = \frac{1}{n}.
\end{equation}
Because $n$ is fixed it is clear that this move obeys detailed balance. Note that since both these moves independently satisfy detailed balance, they can always be performed in any sequence. See section \ref{sec:implementation} for more details.

\subsection{Implementation}\label{sec:implementation}
% This was probably the most substantional amount of work of this project, so I think this should be reflected in the size of this part
% Also here I'm not sure whether to include the failed attempts?

\paragraph{Data Structures}
% triangles have local information -> faster markov chain
% order 4 list
How can one actually save a triangulation inside a computer? There are of course more than one ways to approach this. By far the most common operation of the simulation is performing Markov Chain steps. It therefore makes sense to choose a data structure in which it is easy (i.e. fast) to execute these updates. Thus we choose to store only adjacency information.

In particular, we keep a list of all \verb|triangles|. Then for each \verb|Triangle| we store three labels corresponding to each of their neighbours (neighbours to the \verb|left|, \verb|right| and in the \verb|time|like directions), as well as its \verb|Orientation| (whether it points \verb|Up| or \verb|Down|). Finally, for convenience a list of \verb|order_four| vertices is stored (labelled by the \verb|Triangle| northwest of the vertex). All of this information is stored in a \verb|Universe| structure. In pseudocode the \verb|Universe| is defined as:
\begin{verbatim}
    Universe = {
        triangles: [Triangle],
        order_four: [int],
    }
\end{verbatim}
And the \verb|Triangle| structure is defined as (in pseudocode):
\begin{verbatim}
    Triangle = {
        orientation: Orientation,
        time: int,
        left: int,
        right: int,
    }
\end{verbatim}
Note that we use \verb|int|egers to label \verb|Triangle|s by index in the list of \verb|triangles| stored in \verb|Universe|.

With these data structures the implementation of the moves is relatively straightforward. Uniformly sampling a \verb|triangle| or \verb|order_four| vertex can be done by simply generating a random \verb|int|eger between $1$ and the length of \verb|triangles| or \verb|order_four|, respectively. Sampling a spacelike edge is achieved by sampling a \verb|triangle| and selecting its spacelike edge.

In order to actually execute the shard move, six pairs of neighbours need to be reassigned, and a single \verb|order_four| vertex is moved\footnote{If the order 4 vertices were labelled by the triangles northeast of them, even this would not be necessary. In that case the order 4 vertex would automatically move with the shard to its new location. Unfortunately we only realised this near the very end of the project.}. For the flip move, two pairs of neighbours need to be updated. There are two vertices that might become \verb|order_four|, these should be checked individually. Additionally, there are two that, if they were \verb|order_four| before the move, they certainly no longer are afterwards. These four vertices are marked with green circles in Fig. \ref{fig:flip_move}.

\paragraph{Length Measurements}
% Explain how length profile is measured

\paragraph{Input Parameters}
% explain all relevant input parameters of the cli



\subsection{Observables} \label{sec:observables}
% Explain choice of observables, possibly alternatives that were considered but not measured
% Maybe also explain how lengthprofile is obtained from implementation, as this is not entirely trivial and Timothy asked about this after the presentation
Finding good observables to measure is quite challenging in (C)DT. This may partly be due to a lack of experimental data to compare simulations to.
But also where many Monte Carlo simulations in physics are concerned with describing the behaviour of fields on a lattice, naturally giving some combination of the field values as observables,
there is no such field in CDT but the lattice itself should provide the information of interest.

The chosen observables can of course not depend on any details of the implementation, like the use of labelling in our implementation.
And for an observable to be of physical interest it should also have a well-defined limit for the amount of triangles $N \rightarrow \infty$.
Observables that are often looked at are different notions of dimension, notably the \emph{spectral dimension} \cite{2012} and \emph{Hausdorff dimension} \cite{1998, 2012}, the latter of which has also been discussed in the case of Dynamical Triangulation in the course Monte Carlo Techniques. And a newer observable of interest is a certain notion of \emph{Quantum Curvature} -- a sort of extended notion of Ricci curvature which can be applied to the used triangulations and obeys a proper limit \cite{brunekreef2021}.

However since the focus of this project was more in the implementation of a Markov Chain Monte Carlo simulation, and the measurements and analysis of these observables is somewhat involved, we only consider observables that are derived from the volume of the space-like slices.
This means that in $1 + 1$D CDT we look at the \emph{length}, that is the amount of edges or equivalently the amount of vertices in a time-slice, at different times; this we call the \emph{length profile} $\ell(t)$

\paragraph{Length profile}
The length profile itself still depends on a certain choice of origin in $t$, while we consider a toroidal topology so any there is no preferred origin. Thus, any observable should have some sort of averaging over $t$.

The simplest observable to consider is the average of $\ell(t)$, but this is trivially $L$ since the total amount timeslices and triangles and thus vertices is kept fixed.
So the simplest non-trivial observable we can think of is the standard deviation $\sigma_\ell$ of the length profile $\ell(t)$:
\begin{equation}
    \sigma_\ell^2 = \frac{1}{T} \sum_{t = 1}^{T} \Big(\ell(t) - L\Big)^2,
\end{equation}
using the usual definition of the population variance.

%TODO possibly we want to use a different notion of length correlation
But $\sigma_\ell$ does not contain any information of the relation of lengths between different times $t$.
So an arguably more interesting observable is what we call the \emph{length correlation}:
\begin{equation}
    \rho_\ell(t) = \frac{1}{T \sigma_\ell^2} \sum_{t_0 = 1}^{T} \qty(\ell(t_0) - L)\qty(\ell(t_0 + t) - L),
\end{equation}
where this correlation is normalised such that $\rho_\ell(0) = 1$.
Note that the length correlation is still a function of $t$ but this $t$ is the time difference between two time-slices, so there is no origin dependence.

Both observables $\sigma_\ell$ and $\rho_\ell(t)$ do diverge for $N \rightarrow \infty$ as they will scale with $N$.
How they will scale is at this point not known, but we hope they will scale in a well-behaved way such that with proper rescaling these observables may have a well-defined limit.