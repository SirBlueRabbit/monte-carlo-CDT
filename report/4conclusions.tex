Using Markov Chain Monte Carlo methods, we have been able to sample 2D CDT universes with fixed volume. On these universes we were able to measure the length profile $\ell(t)$. From this length profile we computed the standard deviation $\sigma_\ell$ and the autocovariance $\rho_\ell(t)$. These were compared to a continuum theory for various system sizes at constant $T/L$ ratio.

Both observables seemed to fit the theoretical prediction quite well, though not perfectly. There are a couple of possible sources of errors. Perhaps most importantly there are still significant statistical errors, mainly visible in Fig. \ref{fig:cov_collapsed}. Furthermore, the theory we used is only valid in the limit $T/L \to \infty$. It is of course not possible to reach this limit in practice. One possible future approach could be to further develop the theory to include higher order corrections. Another option is to investigate how the results scale for different $T/L$ ratios; we only looked at the case $T/L = 20$. However, both of these approaches will likely first require a decrease in statistical errors.

Finally, there are many more interesting observable to measure in 2D CDT, some of which are already discussed in section \ref{sec:observables}.
If the time permits it is highly recommended measuring some of these observables, but we could not find the time in this project to look at these.