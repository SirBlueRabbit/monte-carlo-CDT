\subsection{Quantum Gravity}

A field of theoretical physics which still has many open questions is quantum gravity. One may notice that "quantum gravity" consists of two words: "quantum" and "gravity". The "gravity" aspect is quite well understood, and its content can be compactly summarised in the Einstein-Hilbert action
\begin{equation}
    S[g_{\mu \nu}]
    =
    \frac{1}{16 \pi G}
    \int_{\mathcal{M}} \dd[d]{x} \sqrt{-g(x)}
    \qty(R(x) - 2 \Lambda)
    ,
\end{equation}
where the integration is over the entire $d$-dimensional spacetime manifold $\mathcal{M}$. Here $R(x)$ is the Ricci curvature scalar, $g(x)$ is the determinant of the metric tensor $g_{\mu \nu}$, $G$ is Newton's gravitational constant and $\Lambda$ is the cosmological constant. Classical equations of motion can then be obtained via a variational principle by solving $\variation{S} = 0$.

A common approach to add the "quantum" aspect to any classical theory is by introducing a path integral. Since in our case the classical action is a functional of the metric $g_{\mu \nu}$, the path integral is given by
\begin{equation}
    Z
    =
    \int \frac{\mathcal{D}[g_{\mu \nu}]}{\text{Diff}(\mathcal{M})}
    e^{i S[g_{\mu \nu}]}
    .
\end{equation}
But what does this actually mean? How could one explicitly carry out this integration?

\subsection{Causal Dynamical Triangulations}

We somewhat avoid these questions by adopting a certain regularisation approach, namely Causal Dynamical Triangulations (CDT). Instead of trying to integrate over all continuously curved manifolds, we sum over piecewise flat manifolds that are constructed by gluing a certain class of flat building blocks together. In particular, these building blocks will be $d$-dimensional regular simplices with a flat Minkowskian interior. These simplices have squared edge lengths fixed at $\pm a^2$ (allowing space- and timelike edges), where $a$ is also a regularisation parameter. We foliate the manifold into $T$ $(d - 1)$-dimensional timeslices, labelled by $t = 0, \ldots T - 1$, connected by a single layer of $d$-dimensional simplices. Different manifolds can be obtained by changing the way simplices are connected. Now the regularised path integral can be written as
\begin{equation}
    Z
    =
    \sum_{T \in \mathcal{T}} \frac{1}{C(T)} e^{i S[T]}
    ,
\end{equation}
where the summation is over all triangulations. Here $S[T]$ is the action and $C(T)$ is a symmetry factor for a given triangulation $T$\footnote{Unfortunately the symbol $T$ is the most natural choice for denoting both a particular triangulation, as well as the number of timeslices contained in such a triangulation. Hopefully it will be clear from context which of the two is used.}.

\paragraph{Simplifications} To simplify matters, we only consider 2-dimensional spacetimes with periodic boundaries. The resulting topology is that of a 2-torus. Note that in this case each timeslice is just a ring of spacelike edges. The number of spacelike edges in timeslice $t$ is denoted by $\ell(t)$. With this in mind, we make a few useful observations:
\begin{itemize}
    \item All triangles have equal volume, so the total volume is proportional to the number of triangles.
    \item The manifold is piecewise flat, meaning a Wick rotation is well-defined.
    \item In 2 dimensions the integrated scalar curvature is constant and therefore irrelevant for the dynamics.
\end{itemize}
Using these observations we find that the Euclidean action for a 2-dimensional triangulation $T$ is \cite{2012}
\begin{equation}
    S[T] = \lambda N(T),
\end{equation}
where $\lambda$ is a constant related to the cosmological constant $\Lambda$. For now we can just interpret it as a model parameter.

To simplify the symmetry factor $C(T)$ we label all triangles\footnote{This is of course also convenient when encoding the triangulation in a computer.}. Each triangulation can be labelled in $N(T)!$ ways, so the Euclidean path integral (or partition sum) becomes
\begin{equation}
    Z
    =
    \sum_{T_\ell \in \mathcal{T}_\ell} \frac{1}{N(T_\ell)!} e^{- \lambda N(T_\ell)}
    ,
\end{equation}
where the summation is now over all labelled triangulations. It is possible to split the partition sum into contributions with a certain number of triangles, as follows
\begin{equation}\label{eq:part_sum}
    Z
    =
    \sum_{n = 0}^\infty \Omega(n) \frac{e^{- \lambda n}}{n!}
    ,
\end{equation}
where $\Omega(n)$ denotes the number of labelled triangulations with $n$ triangles.

\subsection{Continuum Limit}
% Ik wil naar deze vergelijkingen kunnen verwijzen, ze mogen gerust verandert en verplaatst als ze maar ergens staan
\begin{equation}\label{eq:cov_ell}
    \text{Cov}\big[\ell(0), \ell(t)\big] = \Exp{\qty(\ell(0) - L)\qty(\ell(t) - L)}
    = \frac{1}{2\Lambda} e^{- 2\abs{t}\sqrt{\Lambda}}
\end{equation}
\begin{equation}\label{eq:std_ell}
    \sigma_\ell^2 = \text{Var}\big[\ell(t)\big] = \text{Cov}(t = 0)
    = \frac{1}{2\Lambda}
\end{equation}
\begin{equation}\label{eq:exp_ell}
    \Exp{\ell(t)} = \frac{1}{\sqrt{\Lambda}} = L
\end{equation}

% Describe continuum limit of CDT, mainly include things that will be used as observables later on.

