% Brief overview of the observables that are measured
The observables we measured in the $1 + 1$D CDT model are the \emph{standard deviation} of the length profile $\sigma_\ell$ and the \emph{length correlation} $\rho_\ell(t)$ as introduced in section \ref{sec:observables}.
In this section we will present and discuss the results found for these observables.

\subsection{Pre-analysis}
% Determination of the equilibration time and correlation time
Before the actual measurements can be performed some data analysis need to be done beforehand.

\paragraph{Equilibration}
To be able to take measurements of the wanted observables in the Markov-chain Monte Carlo simulation the system needs to be thermalised. Which is to say that the system should be in a `typical' state, such that the expectation value of an observable at any timestep in the simulation is the same as any other.
We start the system in a non-typical, flat spacetime, so it takes some Markov-chain steps before the system is in equilibrium.
We want to only start measurements after this \emph{equilibration time}, so we need to estimate it to know when to start measuring.

Preferably one used the observable of interest to determine the equilibration time, however it is very difficult to quantify when a function like $\rho_\ell(t)$ has thermalised. So we determine the equilibration time using $\sigma_\ell$.
%TODO continue here%


\subsection{Standard deviation}
% Presentation of the results of the standard deviation of the length profile

\subsection{Length correlation}
% Presentation of the results of the analysis of the length correlation of the length profile

\subsection{Discussion}
% Discussion of the interpretation and validity of the results
% Explaination of the difficulties in determining useful results
% Suggestions on improving these measurements